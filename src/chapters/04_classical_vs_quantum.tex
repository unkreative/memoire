\chapter{Classical vs. Quantum Computing}

\section{Computing Paradigms}
Classical and quantum computing represent fundamentally different approaches:

\begin{itemize}
    \item \textbf{Classical computing} uses bits (0 or 1) as the basic unit of information
    \item \textbf{Quantum computing} uses qubits that can exist in superposition of states
\end{itemize}

\section{Key Differences}
The most significant differences include:

\begin{itemize}
    \item \textbf{Parallelism:} Quantum computers can perform calculations on many possible input values simultaneously through superposition
    
    \item \textbf{Entanglement:} Quantum computers utilize correlated qubits that enable more efficient algorithms
    
    \item \textbf{Probabilistic Results:} Quantum measurements yield probabilistic outcomes, requiring repeated runs for reliable results
\end{itemize}

% Temporarily commenting out problematic image - replace with actual PNG file later
% \begin{figure}[h]
%     \centering
%     \includegraphics[width=0.7\textwidth]{src/images/quantum_vs_classical.png}
%     \caption{Comparison of Classical and Quantum Computing Models}
% \end{figure}

\section{Computational Complexity}
Quantum computers excel at specific problems:

\begin{itemize}
    \item \textbf{Integer Factorization:} Shor's algorithm provides exponential speedup
    
    \item \textbf{Search Problems:} Grover's algorithm offers quadratic speedup
    
    \item \textbf{Simulation:} Quantum systems can efficiently simulate other quantum systems
\end{itemize}

However, quantum computers do not offer universal speedups for all computational problems.

\section{Practical Limitations}
Current quantum computers face significant challenges:

\begin{itemize}
    \item \textbf{Error Rates:} Quantum operations have high error rates requiring error correction
    
    \item \textbf{Decoherence:} Quantum states degrade rapidly, limiting computation time
    
    \item \textbf{Scalability:} Building large-scale, reliable quantum computers remains difficult
\end{itemize}

\section{Complementary Roles}
Classical and quantum computing will likely coexist, with quantum computers serving as specialized accelerators for specific problems while classical computers handle general-purpose computing tasks.