\chapter{Introduction}

The intersection of quantum computing and cryptography represents one of the most significant technological shifts of our era. This thesis explores how quantum computers could transform cryptographic systems and examines emerging solutions.

\section{Research Motivation}

Modern digital infrastructure relies heavily on cryptographic systems designed to resist classical computing attacks. However, quantum computing threatens these foundations through algorithms like Shor's, which can efficiently solve problems that are currently computationally infeasible.

As quantum computing advances, we face both opportunities and challenges:
\begin{itemize}
    \item Existing cryptographic systems may become vulnerable
    \item New quantum-resistant approaches are needed
    \item Quantum technologies offer novel cryptographic methods
\end{itemize}

\section{Research Questions}

This thesis addresses several key questions:
\begin{enumerate}
    \item How do quantum computing principles fundamentally differ from classical computing?
    \item What specific threats does quantum computing pose to current cryptographic systems?
    \item What are the most promising quantum-resistant cryptographic approaches?
    \item What implementation challenges exist for quantum-resistant cryptography?
    \item How might the transition to post-quantum cryptography affect society?
\end{enumerate}

\section{Methodology}

This research combines literature review with analysis of current standards development, focusing on:
\begin{itemize}
    \item Academic research on quantum computing and cryptography
    \item Industry standards and NIST's post-quantum cryptography standardization process
    \item Implementation considerations across different technology sectors
\end{itemize}

\section{Thesis Structure}

The thesis progresses from foundational concepts to specific applications:
\begin{itemize}
    \item Chapter 2 explains fundamental quantum computing principles
    \item Chapter 3 compares classical and quantum computing paradigms
    \item Chapter 4 reviews classical cryptographic foundations
    \item Chapter 5 analyzes quantum computing's impact on current cryptography
    \item Chapter 6 examines quantum-resistant cryptographic approaches
    \item Chapter 7 discusses implementation challenges
    \item Chapter 8 explores future research directions
    \item Chapter 9 considers societal implications
\end{itemize}

This work aims to provide a balanced assessment of the changing cryptographic landscape as we enter the quantum computing era.